\documentclass[parskip=full]{scrartcl}
\usepackage[utf8]{inputenc} % use utf8 file encoding for TeX sources
\usepackage[T1]{fontenc}    % avoid garbled Unicode text in pdf
\usepackage[german]{babel}  % german hyphenation, quotes, etc
\usepackage{hyperref}       % detailed hyperlink/pdf configuration
\hypersetup{                % ‘texdoc hyperref‘ for options
	pdftitle={Entwurf},%
	bookmarks=true,%
}
\usepackage{graphicx}       % provides commands for including figures
\usepackage{csquotes}       % provides \enquote{} macro for "quotes"
\usepackage{scrpage2}
\pagestyle{scrheadings}
\usepackage{float}

%\clearscrheadfoot
\ohead{BPTI: Gruppe 03=\{Niklas Metz, Felix Bachmann\}, Aufgabenblatt 1, WS 2017/18}

\begin{document}
	\section{Manual für die Schaltung von Aufgabe 6 und 7}
		\subsection{Features}
			Die Schaltung verfügt über vier Modi. Je nach Modus variiert das Lauflicht. 
			\begin{table}[H]
				\centering
				\caption{Leuchtsequenzen der Modi}
				\label{my-label}
				\begin{tabular}{l|l}
					Modus &  Leuchtsequenz (leuchtende LEDs)                                                                                                                                                              \\ \hline
					0     & \begin{tabular}[c]{@{}l@{}}keine \\ =\textgreater D10,D11 \\ =\textgreater D9,D12 \\ =\textgreater D8,D13 \\ =\textgreater D7,D14\end{tabular}                                               \\ \hline
					1     & \begin{tabular}[c]{@{}l@{}}keine \\ =\textgreater D13,D14 \\ =\textgreater D11,D12,D13,D14\\ =\textgreater D9,D10,D11,D12,D13,D14 \\ =\textgreater D7,D8,D9,D10,D11,D12,D13,D14\end{tabular} \\
					\hline
					2     & \begin{tabular}[c]{@{}l@{}}keine \\ =\textgreater D13,D14 \\ =\textgreater D7,D8,D13,D14\\ =\textgreater D7,D8,D11,12,D13,D14 \\ =\textgreater D7,D8,D9,D10,D11,D12,D13,D14\end{tabular}       \\
					\hline
					3    & \begin{tabular}[c]{@{}l@{}}keine \\ =\textgreater D7,D8,D13,D14 \\ =\textgreater D8,D9,D12,D13 \\ =\textgreater D9,D10,D11,D12 \\ =\textgreater D10,D11\end{tabular}                        
				\end{tabular}
			\end{table}
			Nachdem der letzte Zustand erreicht ist, beginnen die Modi wieder von vorn.
			Im Folgenden bezeichne ein Schritt eine Änderung der Leuchtsequenz.
			 Es besteht die Möglichkeit die Geschwindigkeit der Leuchtsequenz zu erhöhen oder zu verlangsamen. Die minimale Geschwindigkeit beträgt $\frac{1 Schritt}{4 Sekunden} = 0,25 \frac{Schritte}{Sekunde}$, die maximale Geschwindigkeit beträgt $\frac{1 Schritt}{0,06 Sekunden} \approx 17 \frac{Schritte}{Sekunde}$. Die initiale Geschwindigkeit beträgt $\frac{1 Schritt}{0,5 Sekunden} = 2 \frac{Schritte}{Sekunde}$ Die Geschwindigkeit zu erhöhen hat eine höhere Priorität als die Geschwindigkeit zu verringern.
			Außerdem kann zu jedem Zeitpunkt der Leuchtsequenz die Richtung der Leuchtsequenz umgekehrt werden.
			Zudem können die Modi gewechselt werden, beginnend bei Modus 0. Die Modi haben dann die Reihenfolge: Modus 1, Modus 2, Modus 3, beim vierten Taster-Druck gelangt man wieder zu Modus 0.
			Das Auswählen der Modi hat Priorität gegenüber dem Umkehren der Richtung.
			Zudem kann zu jedem Zeitpunkt ein Reset ausgeführt werden, der die Richtung, Geschwindigkeit und die Modus-Wahl zurücksetzt.
			Der Reset hat Priorität über alle anderen möglichen Aktionen.
			
			\subsection{Bedienkonzept}
				\begin{table}[H]
					\centering
					\caption{Taster-Belegung}
					\label{my-label}
					\begin{tabular}{l|l}
						Taster & Aktion                                  \\
						\hline
						SW4    & Reset                                   \\
						SW5    & Modus auswählen (+1 mod 4 in Tabelle 1) \\
						SW6    & Richtung wechseln                       \\
						SW7    & Geschwindigkeit verringern              \\
						SW8    & Geschwindigkeit erhöhen                
					\end{tabular}
				\end{table}
			
			\subsection{exklusive Features der Schaltung von Aufgabe 7}
				Die Schaltung der Aufgabe 7 ist eine Erweiterung der Schaltung aus Aufgabe 6. Insbesondere funktionieren also alle Features der Schaltung aus Aufgabe 6 auch bei der Schaltung für Aufgabe 7.
				Bei letzterer ist es zusätzlich möglich über die DIP-Schalter SW3.1 bis SW3.8 eine Eingabe zu tätigen. Hierbei leuchtet jede LED, dessen Position im DIP-Schalter auf "ON" gestellt ist bei Muster 3 unabhängig vom aktuellen Zustand der Leuchtsequenz.
	
\end{document}